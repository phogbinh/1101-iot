\documentclass[12pt, a4paper, onside]{article}
\usepackage[affil-it]{authblk} % author institution
\usepackage[backend=biber]{biblatex}

\addbibresource{reference.bib}

\title{\textbf{Internet of Things: Technologies and Applications -- Homework 6}}
\author{Tran Phong Binh\thanks{Student ID: 110062421}}
\affil{Department of Computer Science, National Tsing Hua University}
\date{\today}

\begin{document}

\maketitle

The Internet of Things (IoT) has become more and more incorporated into our modern society. One of the most important applications of this societal transformation is perhaps the massive machine-type communication (mMTC) specifications for 5G, where information is exchanged and processed among millions of devices per square kilometer. Among the existing cellular technologies, long-term evolution machine-type communication (LTE-M) and narrowband IoT (NB-IoT) emerge as two of the most promising candidates for fulfilling the aforementioned requirements. To address mMTC in general and NB-IoT in specific, the Third Generation Partnership Project (3GPP) published the LTE Release 14, which equips NB-IoT with several new features such as increased positioning accuracy, increased peak data rates, the introduction of a lower device power class, improved non-anchor carrier operation, multicast, and authorization of coverage enhancements. The article \cite{nbiot} surveys this topic to a great extent, beginning with a review of LTE Release 13 -- the standardization of NB-IoT, and pushing on with the discussion of the 14th Release.

NB-IoT is a cellular radio access technology supporting low-power wide-area (LPWA) network under licensed spectrum, whose standards in Release 13 were those typical for MTC: long device battery life, low device complexity (which leads to low cost), capability of handling a massive number of devices, and coverage enhancements so as to access user equipments (UEs) underground and at other challenging locations. NB-IoT comprises three deployment options: in-band (deployed inside an LTE carrier), guard band, and standalone (deployed in a designated spectrum). In all operation modes, the specification offers up to 20dB enhanced coverage when compared to the general packet radio services (GPRS) system, which is materialized by repetitions in time and power boosting in in-band and guard band modes.

To be compatible with existing LTE equipment and software vendors, NB-IoT inherits several of LTE designs, including numerologies, channel modulation and coding schemes, and higher layer protocols. Notably, orthogonal frequency-division multiple access (OFDMA) is adopted for downlink communications and single-carrier frequency-division multiple access (SC-FDMA) for uplink. With that said, there are also major differences between NB-IoT and legacy LTE's specifications, with the most significant one being the two technologies' uplink. A subcarrier spacing of 3.75 kHz is introduced in NB-IoT, in addition to the 15 kHz used in legacy LTE systems. Furthermore, resource allocations with less than twelve 15-kHz-subcarriers in the uplink are proposed.

To reduce the device cost, NB-IoT has the following adaptations: uplink and downlink bandwidth of 200 kHz is supported; only half-duplex frequency-division duplex is granted access; only one receiving antenna is required; quadrature phase shift keying (QPSK) and convolutional code are adopted for downlink (less demanding than LTE's turbo code), with single-tone transmissions by pi/2-binary phase shift keying (BPSK) and pi/4-QPSK for uplink (reducing peak-to-average power ratio); only one adaptive asynchronous hybrid automatic repeat request (HARQ) process is supported for both uplink and downlink; the maximum transport block sizes (TBSs) are shortened to 680 bits for downlink and 1000 bits for uplink.

Albeit 3GPP NB-IoT Release 14 presents numerous advancements, we will only present the new features of the publication in this report (as others' descriptions can be referred to the previous release mentioned above): positioning, multicast, new UE output power class. There are two goals for the standardization of NB-IoT positioning: to develop support for the feature of observed time difference of arrival (OTDOA); and to complete UE measurement requirements for enhanced cell identity (CID), which involve evolved Node B (eNB) Rx-Tx time difference, reference signal received power (RSRP), and reference signal received quality (RSRQ). Specifically, OTDOA is a downlink-based positioning method, where a device measures the times of arrivals (TOAs) of positioning reference signals (PRSs) received from multiple transmitting nodes relative to a reference node's PRS transmission to form the reference signal time difference (RSTD) measurements. These measurements yield geographical hyperbolas, whose intersection is determined to be the estimated UE position -- a mechanism taken from Global Positioning System (GPS).

Multicast is another newly developed functionality in NB-IoT -- a group communication where data is transmitted to a group of users in a single transmission. Two applications of paramount importance are: firmware update (e.g. upgrade multiple sensors protocol) and simultaneous control (e.g. turn street lamps on or off). In Release 14, multimedia broadcast multicast services (MBMS) is enabled through single-cell point-to-multipoint (SC-PtM). The SC-PtM data is carried on the single-cell MBMS traffic channel (SC-MTCH) logical channel, which is mapped to the Narrowband Physical Downlink Shared Channel (NPDSCH) similar to unicast traffic, and the scheduling of SC-PtM in NB-IoT follows suit. Since the same TBS and coverage enhancement settings can be used, downlink channel capacity similar to unicast can be achieved, where the maximum number of different supported SC-MTCHs is 64.

The third addition to NB-IoT is a new low UE power class that aims for smaller form factor, lower device power consumption, and lower device cost in comparison with the 20 dBm devices proposed in Release 13, which makes possible a system on chip design with an integrated power amplifier. In specific, the maximum output power is diminished to 14 dBm, reducing the drain current and wasteful voltage drop substantially. Nevertheless, this drives the need to increase the uplink transmission time in order to maintain the respective coverage. But even this resolution comes with a tradeoff: the uplink resource utilization is worsen, and downlink control signaling is increased. To tackle this issue, maximum coupling loss (MCL) for low-power devices are relaxed compared to the previous NB-IoT release. For instance, if we consider a coverage relaxation of 9 dB, devices experiencing a coupling loss above 164-9=155 dB to the serving cell are excluded from the system.

In conclusion, this is a brilliant overview on NB-IoT and its updates in 3GPP Release 14, and I do recommend scholars of interest to take on a thorough research of it.

\printbibliography

\end{document}
