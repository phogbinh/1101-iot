\documentclass[12pt, a4paper, onside]{article}
\usepackage[affil-it]{authblk} % author institution
\usepackage[backend=biber]{biblatex}

\addbibresource{reference.bib}

\title{\textbf{Internet of Things: Technologies and Applications -- Homework 8}}
\author{Tran Phong Binh\thanks{Student ID: 110062421}}
\affil{Department of Computer Science, National Tsing Hua University}
\date{\today}

\begin{document}

\maketitle

Positioning and localization play an important role in modern society, especially precise position location, where the coordinates of objects are estimated in centimeter-level accuracy. For instance, demands for accurate positioning of machineries and products in automated factories are rising due to the rapid development of Internet of Things (IoT), in addition to geofencing applications that monitor people, objects, vehicles, determining whether users entering/leaving a room, or tracking people in hospitals, factories, within and outside buildings. Although the current technologies in Long-Term Evolution (LTE) signaling and Global Positioning System (GPS) can give guaranteed outdoor accuracy of 5 meters, in indoor obstructed environments, or in underground parking areas and urban canyons, they are unreliable as GPS signals are attenuated and reflected. The article \cite{localization5g} does an excellent survey of promising resolutions to this problem that leverage vast mmWave spectrum and narrow beam antenna technology in 5G.

The authors first address the problem by proposing non-line-of-sight (NLoS) mitigation methods. To overcome the poor performance of traditional time of arrival (ToA), time difference of arrival (TDoA), and angle of arrival (AoA)-based localizations in NLoS scenarios, NLoS mitigation techniques aim to identify and ignore NLoS signals to only perform positioning with LoS data. By experiments, one proposal observes that with conventional WiFi radios operating at ISM radio band, the AoA varies by more than 5 degrees when the user equipment (UE) is moved by 5 centimeters under NLoS environment, and as a consequence designs systems to classify such signals to be NLoS and discard them from use in estimating position, achieving a localization accuracy of 23 centimeters with 6 WiFi access points. Another method is to identify base station (BS)-UE NLoS links based on the running variance of the BS-UE distance estimates: If the running variance is greater than a calibrated threshold, then the signal is classified as NLoS and subsequently taken out of consideration. To account for the movement of mobile UEs, a constant proportional to the square of one's velocity is added to the threshold. The third approach utilizes channel features such as maximum received power, root mean square (RMS) delay spread, Rician-K factor, and the angular spread of departure/arrival to carry out the NLoS classification. To be specific, NLoS channels have lower maximum received power of power delay profile (PDP) because there are obstructions and reflectors presented, higher delay spread, close-to-zero Rician-K factor (known to describe degree of multipath in a signal), and wider angular spread (as multipath components arrive from varied directions). The last proposed method is support vector machine (SVM), which outperforms individual features, reducing the NLoS identification error rate from 10 percent to 5 percent.

The author continues by pointing out the issues of the aforementioned methods: as NLoS signals are discarded, these systems waste multipath signal energy, and require dense BS deployment -- the UE must be in LoS of two or more BSs for classical LoS positioning techniques to work, which directly leads to high costs. Four alternatives are introduced in the article: cooperative localization, machine learning for localization, user tracking and data fusion, and localization algorithms exploiting multipath. Since the first two paradigms align with my future research on 5G, I shall illustrate them in this report. With the introduction of device-to-device (D2D) protocols in 5G, UEs may now communicate directly with one another instead of relying solely on base stations to construct localization of the system. There are two main categories of cooperative localization presented: centralized and distributed. In centralized localization, relative UE positioning (range and angular) measurements are sent to one of the serving BSs or a central server for processing, after which positions of all the UEs in the network are simultaneously obtained using nonlinear least squares (LS) estimation. However, it is evident that such centralized processing may result in network congestion in large IoT networks. In distributed systems, UEs are localized based on local measurements exchanged by neighboring nodes. The location estimates of the UEs are then iteratively refined until all neighboring UEs reach a mutual consent. Experiments demonstrate that such a distributed cooperative positioning approach can achieve a root mean square error of 3 meters in an indoor environment over an area of approximately 40 meters x 20 meters with 4 BSs with known locations and 13 unknown UE locations. Secondly, machine learning can be applied to solve the localization problem by building a fingerprinting database containing channel parameters (received signal strength (RSS), channel state information (CSI), AoA of the strongest signal of all BS links)  measured at each reference point with its coordinates. In the real-time online position location phase, the BS-UE channel is measured by the UE, the result of which is matched to the fingerprinting database stored in the UE or in the network to estimate the UE position. There are many options for implementing matching: maximum a posteriori (MAP) estimation, k-nearest neighbor (k-NN), or neural net.

In conclusion, I found the paper insightful and informative, and can serve as one of my research guidelines in the upcoming years.

\printbibliography

\end{document}
