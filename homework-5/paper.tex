\documentclass[12pt, a4paper, onside]{article}
\usepackage[affil-it]{authblk} % author institution
\usepackage[backend=biber]{biblatex}

\addbibresource{reference.bib}

\title{\textbf{Internet of Things: Technologies and Applications -- Homework 5}}
\author{Tran Phong Binh\thanks{Student ID: 110062421}}
\affil{Department of Computer Science, National Tsing Hua University}
\date{\today}

\begin{document}

\maketitle

Long Range Wide Area Network (LoRaWAN) is well known in Internet of Things (IoT) applications for its far-reaching scope and modest energy consumption: its gateway can cover kilometers of range, all the while supporting thousands of devices simultaneously. As a result, it has been enabling precision agriculture, smart cities, remote environment monitoring, infrastructure monitoring, and smart grids. However, since LoRaWAN only supports one-hop communication between end nodes and gateways, it is extremely susceptible to shadowing (signals being `worn out' by medium obstacles such as buildings, hills, etc.) and co-channel interference (signals being disrupted by other wireless devices using the same frequency spectrum) when it comes to distant communications with its sensors. To tackle this issue, several research on routing schemes have been conducted to create multi-hop communication for LoRaWAN, boosting the adaptability of the network in miscellaneous IoT scenarios.

The article \cite{loraRoute} is recognized to be the first one to review extensively on routing protocols proposed for LoRa systems, presenting us with comprehensive insights of ingenious ideas of the field of study. The paper begins with a technical overview of LoRaWAN. Departing from what has been stated in this report, it is said that a LoRa transmission requires determining four values: spreading factor (SF), carrier frequency, bandwidth, and coding rate, with the first one being the paramount factor in the discussion of routing protocols. SF is an integer value from 7 to 12 (the lower the value, the higher the data rate). Other technical specifications of LoRaWAN comprise of two physical layer implementations (frequency shift keying and chirp spread spectrum), `star-of-stars' network topology, three end node types (A, B, and C), and two device activation mechanisms (activation by personalization and over-the-air activation).

The authors continue by describing the two prime routing paradigms adopted in LoRaWAN: tree topology, which explicitly searches for a route starting at the root through a tree hierarchy, and flooding, where the packet received is forwarded without looking for specific routes. There are five tree protocols surveyed: LoRa Mesh, Tree-Based SF Clustering Algorithm (TSCA), Routing Protocol for Low Power and Lossy Networks (RPL), synchronous LoRa mesh, and destination-sequenced distance vector (DSDV). The idea of LoRa Mesh is pretty much the same as that of breadth-first search (BFS). A LoRa gateway is selected as the root of the tree, from which beacons are broadcasted. Sensors that hear the beacons add themselves as children of the gateway by replying a JOIN message. Sequentially, these one-hop-count children then send out beacons to those unreachable by the gateway (maybe because they are shadowed by obstacles), having them as their children with a hop count of two. The process is repeated until all devices are appended into the network. On the other hand, TSCA initiates by creating a tree rooted at the gateway where all nodes in the network use SF7 (the lowest SF value i.e. the highest data rate). Then, the protocol generates subtrees (each having the gateway as its root) of different SF values by the following objective for each: balancing the number of hops from a subtree's vertex to its root and the SF value e.g. subnetworks with high SF should have fewer hops than those with small SF. In terms of RPL, it creates a directed acyclic graph rooted at the gateway. RPL provides upward and downward routes, depending on the destination (the root or the nodes), in which a distance vector algorithm is employed for route selection. The fourth proposal -- synchronous LoRa mesh -- is designated to support scenarios where sensor nodes (SNs) transmit their data and relay information from other SNs located underground. The protocol fabricates a communication cycle with time-division multiple access (TDMA), where the router node (root of the tree) assigns time slots for SNs to listen. Synchronization is done by sending beacons in different hops through the whole network. Lastly, DSDV has every end devices save a routing table containing 5 fundamental information: all destination reachable from that vertex, the next vertex in a path, the number of hops to the root i.e. the gateway, a sequence number, and a timestamp to dispose old packets. A special feature of this network is that it only transports information in the upward direction.

Three flooding approaches are then briefly described in the literature: Concurrent Transmission protocol applied to LoRa nodes (CT-Lora), method of Abrardo and Pozzebon, and LoRaBlink. The first approach sets one node to be the source of data, with the remainder being the re-transmitters. CT does not use any collision avoidance mechanism, but depends on the capture effect to detect at least one of the forwarded packets. In contrast, Abroardo and Pozzebon suggest a linear sensor network (LSN) for underground setups with a total length of 15km e.g. medieval aqueducts in Siena, Italy. In this architecture, every sensor must transmit and receive only to its immediate neighbors i.e. line topology, complying with the schedule of three phases: synchronization, data, and sleep. The final review is on LoRaBlink, which uses time slots for transmission and reception (with an assumption that all participants of the network are synchronized). The concept is straightforward: The gateway sends beacons out to all the devices, which must be transmitted back (not in a sequential order), so that the gateway can estimate how many hops it must execute to reach each destination. After this initialization, the sensors can exchange data with the gateway by asymptotically optimal relaying.

In conclusion, this is an excellent study on routing schemes for LoRaWAN, and I do recommend researchers and practitioners of IoT to grind into it.

\printbibliography

\end{document}
