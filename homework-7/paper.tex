\documentclass[12pt, a4paper, onside]{article}
\usepackage[affil-it]{authblk} % author institution
\usepackage[backend=biber]{biblatex}

\addbibresource{reference.bib}

\title{\textbf{Internet of Things: Technologies and Applications -- Homework 7}}
\author{Tran Phong Binh\thanks{Student ID: 110062421}}
\affil{Department of Computer Science, National Tsing Hua University}
\date{\today}

\begin{document}

\maketitle

As the Internet of Things (IoT) is becoming more and more integrated into our day-to-day lives, unmanned aerial vehicles (UAVs) emerge to be one of the most paramount applications which transform how information is collected and transferred in the near future. Albeit a single UAV can sometimes handle missions like military scouting, goods freight in cities, and road traffic management, multiple UAVs are often employed as such systems provide faster multitasking ability, longer network lifetime, and higher scalability. Nevertheless, the employment of multi-UAVs introduces the demand to cooperate UAVs' communications efficiently; that is, routing protocols for UAVs need to be studied and properly applied. The article \cite{uav} provides a comprehensive overview of the research area, differentiating traditional routing protocols from those designed for UAV ad hoc networks (UANETs), and wrapping up with open research questions of the field. In this report, I shall give a quick review of five main categories of routing protocols for UANETs: single-hop routing, proactive routing, reactive routing, hybrid routing, and position-based routing, as it is arguably the most important topic of the research and intimately correlated with my future thesis -- 5G network scheduling.

Since existing routing protocols in mobile ad hoc networks (MANETs) and vehicle ad hoc networks (VANETs) do not account for high mobility of UAVs, drastically changing network topology and intermittently connected communication links require to be addressed, they cannot be directly applied onto UANET scenarios. Thence, many efforts have been made to design UAV-specific routing schemes, divided into two main paradigms: single-hop routing and multi-hop routing. The former uses a static routing table i.e. it does not update to transmit packets in one hop: a UAV first loads packets from a source, then directly transfers the packets to a destination without another forwarding middleman. Obviously, this kind of routing protocol is light-weight and chiefly designed for fixed topology networks, implying that it performs poorly when it comes to fault tolerance and dynamic environments. Two notable single-hop protocols are discussed: load-carry-and-deliver (LCAD) and differential evolution with particle swarm optimization (DEQPSO). LCAD comprises three phases: loading packets from the source node, carrying packets when flying, and delivering packets to the destination node. It grants higher network throughput over conventional multihop stored-and-forwarding routing protocols as UAV's direct packet delivery bypasses interference and medium access contention, but experiences severe latency in exchange. DEQPSO aims to solve the route planning problem -- the main issue in static routing protocols -- by combining DE and QPSO to reduce the detection probability by radar. Experiments demonstrate that DEQPSO can not only achieve better path planning compared to Voronoi diagram, DE, and QPSO, but also speed up the convergence of its two inner algorithms.

In contrast with the first paradigm, multi-hop routing protocols forward packets hop by hop, prompting the need for a capable next-hop node selection algorithm. Based on different selection strategies, the paradigm is classified into two sets: topology-based (which is further segmented into three categories: proactive routing, reactive routing, and hybrid routing) and position-based routing. In proactive routing, routing information is recorded and stored in each UAV in advance, enabling immediate routing path selection for packet transmission. However, this group of protocols suffers from high communication overhead caused by a large amount of control packets required for route establishment. Topology broadcast based on reverse-path forwarding (TBRPF), directional optimized link state routing (DOLSR), speed-aware predictive-OLSR (POLSR), and mobility and load aware OLSR (ML-OLSR) are surveyed in details, with a remark on multipoint relay (MPR) utilized in OLSR protocols that greatly reduces message overhead. Reactive routing protocols are on-demand routing oriented; that is, it finds a routing path only when packets are scheduled for transmission. This effectively reduces the control message overhead, but adds up end-to-end delay. Four representatives for this approach are proposed: dynamic source routing (DSR), reactive-greedy-reactive (RGR) based on ad hoc on-demand distance vector (AODV), Modified-RGR, UAV-assisted routing (UAVR). The last protocol is of my most interest for its efficiency and robustness against fast moving UAVs. It is a bi-phase framework: UVAR-G for ground-to-UAV communication and UAVR-S for UAV-to-UAV communication. In the first phase, packets are forwarded to UAVs through multihop routes, whilst dynamic routing is discovered to further forward packets in the latter phase (if a path is disconnected due to UAV's movement, an alternative path is found). Hybrid routing protocol is simply a combination of proactive and reactive routing protocols, which is intended to overcome the problem of high control message overhead in the first and the long end-to-end delay in the second. The last resolution -- position-based routing protocols -- utilizes geographic location information, among which the prominent candidate is recovery strategy greedy forwarded failure (RSGFF), which tries to tackle the problem GPSR faces in sparse environments: if no neighbor node is closer to the destination than it is, the algorithm fails. It contains of three sub-strategies: wait and retry once, forward the packet to the furthest neighbor node, forward the packet to the best-moving node -- the neighbor node moving faster towards the destination.

In conclusion, the paper is a brilliant description of the current state-of-the-art routing protocols for UANETs, and I do recommend researchers of interest to have a thorough read of it.

\printbibliography

\end{document}
