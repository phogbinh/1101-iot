\documentclass[12pt, a4paper, onside]{article}
\usepackage[affil-it]{authblk} % author institution
\usepackage[backend=biber]{biblatex}

\addbibresource{reference.bib}

\title{\textbf{Internet of Things: Technologies and Applications -- Homework 3}}
\author{Tran Phong Binh\thanks{Student ID: 110062421}}
\affil{Department of Computer Science, National Tsing Hua University}
\date{\today}

\begin{document}

\maketitle

Since the invention of IEEE 802.11, it has been playing a vital role in changing the life of millions on the planet, most notably the recent deployments of IEEE 802.11n (Wi-Fi 4) and 802.11ax (Wi-Fi 5) whose data rates are up to 450Mbps and 1.73Gbps, respectively. For their mobility, flexibility, and ease of use, the technology has been utilized in regions of multiple overlapping basic service sets (OBSS). The article \cite{wifi6} gives a comprehensive introduction to IEEE 802.11ax -- the promising Wi-Fi 6, with a highlight in the advanced technological leaps proposed to improve the efficiency within high density wireless local area networks (WLAN).

The authors begin with an overview of the popularity of wireless networks across the globe, stating that by 2019, the global data traffic will be 10 times higher than the level measured five years before. Narrowing the scope down to indoor applications, IEEE 802.11 based WLANs stand out as one of the most well-known and successful wireless solutions, currently powering countless medium to large scale enterprises, public area hot-spots, apartment complexes, etc. with staggering figures: the global value of the Wi-Fi market was recorded to be 14.8 billion in 2015, and was estimated to double in 2020.

Albeit the latest Wi-Fi 4 and 5 have been doing us good in terms of peak aggregate multi-station throughput, it is argued that we can do better in terms of reducing interference and improving spatial reuse. In specific, the former issue has not been properly addressed, with the latter being capped by the overly protective channel access method in the aforementioned standards. To counteract, Wi-Fi 6 intelligently applies techniques that would increase the physical bit rate, but also reduce the frame error rate (FER) and improve spectral reuse by allowing highly efficient multi-user access and by mitigating/reducing interference, hence boosting area throughput.

As the main topic of this work, key technological features in Wi-Fi 6 are thoroughly analyzed by four categories: PHY, MAC, multi-user, and other notable features. I choose to summarize MAC layer enhancements here, as this is what our course concentrates on, not technologies in the PHY layer. As claimed, Task Group 802.11ax (TGax) is working on two major enhancements for the MAC layer: improving spatial reuse and managing interference. The first section consists of upgrades on four different areas: PHYCCA modifications, transmit power control (TPC), BSS color, and multiple NAVs for spatial reuse. Previous versions of IEEE 802.11 used physical clear channel assessment (PHYCCA) modules to sense whether the channel is busy or idle by measuring the received energy. Wi-Fi 6 aims to formally employ dynamic PHYCCA modifications, which enable multiple concurrent transmissions, increasing spectral reuse. The motivation is that for a closed environment with lots of devices, stations will always sense that the channel is busy (because the carrier sensing range is fixed), even though multiple simultaneous transmissions are viable. The dynamic sensitivity control (DSC) algorithm is proposed to tackle this problem. It works by finely tuning the carrier sense threshold (CST) for each node in a distributed manner. In this way, extremely aggressive and conservative behavior of a station in a bounded area can be avoided. Experiments show that the throughput gains achieved by this technique are more than 20 percent on average when combined with optimal channel selection (gain increases beyond 40 percent when stations use slow bit rates and send long frames). In terms of transmit control power, TGax sets the goal to dynamically adjust the lowest possible power for stations with the highest path loss, aiming to achieve a signal to interference plus noise ratio (SINR) that is sufficient to decode the received frames. In addition, the TPC method in IEEE 802.11ax also constitutes the change of transmit power control of non-AP stations based on the RSSI of beacon signals received from the associated AP. The third enhancement to be made on spatial reuse comes from the idea of BSS color -- a scheme proposed for the IEEE 802.11ah standard that increases throughput of dense WLAN networks. Here, each BSS is assigned a specific color by which a station can identify whether a frame is from a neighboring BSS, allowing it to abandon the reception process i.e. to assume that the channel is idle during that transmission, increasing transmission opportunities. Building up the stack of improvements is the novel proposal of two network allocation vector (NAV) timers at each station -- one being an intra-BSS NAV and the other regular NAV. Technically, the intra-BSS NAV is reset or increased only by the frames from that BSS, allowing the station to ignore request to send/clear to send (RTS/CTS) frames transmitted from the OBSS and expand spatial reuse. On the other hand, to resolve interference in dense deployments, Wi-Fi 6 utilizes the RTS/CTS method based on observed channel conditions on a per node basis; that is, an access point (AP) could use novel mechanisms to remotely set up RTS/CTS for any of it associated stations, which can further mitigate the problem of hidden nodes thwarting transmissions. The article demonstrates a 40 percent throughput gain using DSC and a significant 60 percent throughput gain using DSC combined with an intelligent four-way handshake mechanism in the worst case environment scenario where each station was continuously sending frames of maximal duration.

In conclusion, it was an interesting read on IEEE 802.11ax, covering the rationales behind developing this amendment, use cases, main technological advancements, and the two major challenges the next generation of Wi-Fi networks will face: coexistence with unlicensed LTE, and adoption of the IoT paradigm.

\printbibliography

\end{document}
